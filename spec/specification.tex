\documentclass[10pt]{artikel1}
\usepackage[a4paper,
            bindingoffset=0in,
            left=1.5in,
            right=1.5in,
            top=1in,
            bottom=1in,
            footskip=.25in]{geometry}

\usepackage{syntax}
\usepackage{amsfonts}
\usepackage{amsmath}

\title{Specification for Tost}
\author{}

\newcommand{\eempty}{$\varepsilon$}

\begin{document}
\maketitle

\section*{Lexical Rules}

\begin{enumerate}
    \item Identifiers may be any Unicode alphanumeric character or `\verb|_|' optionally followed by more alphanumeric characters, `\verb|_|', or digits
    \item Line-comments start with `\verb|//|' and go to the end of the line
    \item Block-comments start with `\verb|/*|' and go until a matching `\verb|*/|'. They may nest.
    \item Strings are surrounded by `\verb|"|'. They may contain the following escape sequences \begin{enumerate}
              \item `\verb|\n|' : character newline (ASCII code point 10)
              \item `\verb|\r|' : carriage return (ASCII code point 13)
              \item `\verb|\t|' : horizontal tab (ASCII code point 9)
              \item `\verb|\"|' : literal " (ASCII code point 34)
              \item `\verb|\u{|' codepoint `\verb|}|' : the Unicode character with the given code point, in hexadecimal
          \end{enumerate}

    \item Format strings are start with by `\verb|f"|' and end with `\verb|"|'. They may contain the same escape sequences as normal strings. Additionally, they may contain expressions that are surrounded by `\verb|{|' and `\verb|}|'. Due to current restrictions, these expressions cannot contain `\verb|{|' or `\verb|}|'.

\end{enumerate}

\section*{Grammar}
\setlength{\grammarparsep}{20pt plus 1pt minus 1pt}
\setlength{\grammarindent}{12em}

In the following grammar specification, $\langle \text{UPPERCASE} \rangle$
denotes non-terminal rules, whereas $\langle \text{lowercase} \rangle$ denotes a token. `\verb|literal|' denotes a simple token

Every comma separated list may be empty. Trailing commas are supported.

\begin{grammar}


    <PROGRAM> ::= ( <ITEM> )*

    <ITEM> ::= <FUNCTION>

    <FUNCTION> ::= `toaster' <ident> <HAPPINESS> \\
    <ARG_LIST> <RETURN_TYPE> \\
    <BLOCK>

    <HAPPINESS> ::= `:>' \alt `:<'


    <ARG_LIST> ::= <ident>$_1$ `:' <TYPE>$_1$ `,' \dots `,' <ident>$_n$ `:' <TYPE>$_n$

    <RETURN_TYPE> ::= `->' <TYPE>
    \alt \eempty

    <TYPE> ::= <ident>
    \alt `(' <TYPE> `)'
    \alt `(' <TYPE>$_1$ `,' \dots `,' <TYPE>$_n$ `)'


    <BLOCK> ::= `{' (<STATEMENT>)* <EXPR>? `}'

    <STATEMENT> ::= `return' `;'
    \alt `return' <EXPR> `;'
    \alt `let' <PATTERN> `=' <EXPR> `;'
    \alt <IF_EXPR>
    \alt <RHS> `=' <EXPR> `;'

    <EXPR> ::= <integer>
    \alt <string>
    \alt <EXPR> <BINOP> <EXPR>
    \alt <UNOP> <EXPR>
    \alt <FUNC_CALL>
    \alt `(' <EXPR> `)'
    \alt `(' <EXPR>$_1$, \dots, <EXPR>$_n$ `)'
    \alt <IF_EXPR>

    <IF_EXPR> ::= `if' <EXPR> <BLOCK> \\ ( `else' `if' <EXPR> <BLOCK> )* \\( `else' <BLOCK> )?

    <PATTERN> ::= <ident>

\end{grammar}

\end{document}
